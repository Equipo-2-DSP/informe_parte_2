\documentclass[letterpaper]{article}
\usepackage[spanish,es-tabla]{babel}
\usepackage{indentfirst}
\usepackage{float}
\usepackage[utf8]{inputenc}
\usepackage{graphicx}

\begin{document}
    

\section{Ejercicio 4}

\subsubsection*{Inciso a)}
\begin{equation}
    x(t)=u(t-3)-u(t)
\end{equation}
Utilizando la definción de la CTFT:
\begin{equation}
    X(j\Omega)=\int_{-\infty}^{\infty} x(t)e^{-j\Omega t}dt
\end{equation}
Y analizando que la señal $x(t)$ resulta ser un cajon de amplitud negativa en el intervalo del tiempo $0\leq t \leq 3$:
\begin{equation}
    X(j\Omega)=\int_{0}^{3} -1e^{-j\Omega t}dt
\end{equation}
\begin{equation}
    X(j\Omega)=\frac{1}{j\Omega} \left[e^{-j\Omega t} \vert_{3}^{0}\right] = \frac{e^{-3j\Omega} - 1}{j\Omega}
\end{equation}
Si acomodamos la ecuación anterior llegamos a:
\begin{equation}
    X(j\Omega) = -3e^{\frac{-3j\Omega}{2}} sinc(3F)
\end{equation}
\subsubsection*{Inciso b)}
Para el cálculo de la \textit{DFT} se muestrea la señal $x(t)$ utilizando una frecuencia de muestreo $F_s=1Hz$ con la intencíon de obtener 4 muestras del intervalo distinto de cero de la señal original. El resultado de
este muestreo es igual a la ecuación \ref{S.disc} pero invertida en amplitud. Luego, aplicamos la ecuación \ref{DFT} utilizando $N=4$ con el fin de utilizar todas las muestras distintas de cero:
\begin{equation}
    X[k]=\sum_{n=0}^{3}-1e^{-j\omega_k n} = \frac{1 - e^{-j4\omega_k}}{1-e^{-j\omega_k}} = e^{-j\frac{3}{4}\pi k} \frac{sin(\pi k)}{sin(\pi k/4)}
\end{equation} 

Si hacemos un analisis de las ecuaciones utilizadas para el calculo de la DFT y la DFTF, siendo la ecuacion de la DTFT: 
\begin{equation}
    \label{DTFT}
	\tilde{X}(e^{j \omega T}) = \sum_{n=- \infty}^{\infty} x_{c}(nT) e^{-j \omega nT}=\frac{1}{T}\sum_{m=- \infty}^{\infty} X_{c} \left(j \omega - j \frac{2 \pi}{T}	m \right)	
\end{equation}
 donde $x_{c}(nT)$ son las muestras de $x_{c}(t)$.
 Con los resultados obtenidos del analisis de una señal aperidioca discreta, se obtuvo que la relacion entre la DTFT y la DFT viene dada por el muestreo en frecuencia de la DFTF por parte de la DFT, la relacion
se ve en la ecuacion (omega.k creo que era Xavi). Reemplazando dicha relación en la ecuacion (\ref{DTFT}) se obtiene la expresion de la DFT :
 \begin{equation}
 X[k] = \frac{1}{T}\sum_{m=- \infty}^{\infty} X_{c} \left(j \frac{2 \pi k }{N} - j \frac{2 \pi}{T}	m \right)    k =0,1..., N-1 \\
 \end{equation}
 quedando entonces expuesta la relacion entre ambas.
 
 Si ahora queremos analizar la relacion existente entre la DTFT y la CTFT, solo basta con realizar un nuevo analisis de la ecuacion (\ref{DTFT}), en ella se puede ver implicita la relacion buscada.
 Por inspeccion de la ultima expresion, vemos que la DFTF de la señal muestrada no es mas que la periodización del espectro de la señal continua original.
 En resumen la relacion entre la DTFT y la CTFT es que la DTFT es un espectro periodico con posible aliasing de la señal CTFT.
 

\end{document}