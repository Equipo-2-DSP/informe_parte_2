\documentclass[11pt,a4paper]{article}
\usepackage[utf8]{inputenc}
\usepackage[spanish,es-tabla]{babel}
\usepackage{amsmath}
\usepackage{amsfonts}
\usepackage{amssymb}
\usepackage{graphicx}
\usepackage{float}
\usepackage{hyperref}
%\usepackage{subfigure} 
\usepackage{amsmath}
\usepackage{amsfonts}
\usepackage[usenames]{color}
\usepackage{subcaption}
\usepackage[left=3cm,right=2cm,top=3cm,bottom=3cm]{geometry}
\author{Gatica, Isaias \\ Martin, Santiago \\ Saez, Lautaro Ándres \\ Vidman, Xavier Harry}
\title{Procesamiento digital de señales: estimación del espectro de una señal aperiódica}
\date{}
\begin{document}
\maketitle
\section{Análisis de una señal aperiódica y discreta}
En esta sección se analiza el uso de la DFT para estimar el espectro de una señal aperiódica y discreta. Para ello, se considera en primera instancia la siguiente señal:

\begin{equation}
    x[n]=\delta[n]+\delta[n-1]+\delta[n-2]+\delta[n-3]
    \label{eq.xn}
\end{equation}

Para comenzar el estudio se calcula la DTFT y la DFT de la ecuación (\ref{eq.xn}), a fin de realizar una comparación entre los resultados, y poder concluir si es posible utilizar el algoritmo de la DFT para obtener de forma exacta o aproximada la DTFT de una señal aperiódica.

    \subsection{Cálculo de la DTFT y la DFT}

    Utilizando la definción de la DTFT\footnote{Dimitris Manolakiv and Vinay Ingle - ``Applied Digital Signal Processing'' 5th Edition.}:
    \begin{equation}
        X(e^{j\omega})=\sum_{n=-\infty}^{\infty}x[n]e^{-j\omega n}
    \end{equation}
    Donde:
    \begin{equation}
        \label{omega}
        \omega=2\pi f
    \end{equation}
    Se llega a que la DTFT de la ecuación \ref{eq.xn} es:
    \begin{equation}
        X(e^{j\omega})=1+e^{-j\omega}+e^{-2j\omega}+e^{-3j\omega}
        \label{DTFT.R}
    \end{equation}
    Siguiendo con el análisis, se calcula la DFT utilizando la siguiente definición\footnote{Idem 1.}:
    \begin{equation}
        X[k]=\sum_{n=0}^{N-1}x[n]e^{-j\omega_k n} \qquad , 0 \leq k \leq 7
    \end{equation}
    Donde:
    \begin{equation}
        \label{omega.k}
        \omega_k=\frac{2\pi k}{N}
    \end{equation}
    Obteniendo como resultado:
    \begin{equation}
        \label{DFT.R}
        X[k]=1 + e^{-j \frac{\pi}{4} k} + e^{-j \frac{\pi}{2} k} + e^{-j \frac{3 \pi}{4} k}
    \end{equation}
    
    Al cotejar los resultados obtenidos, se observa que las expresiones son muy similares. La diferencia recae en las frecuencias de las exponenciales. La DTFT expresada en la ecuación (\ref{DTFT.R}) tiene como variable una frecuencia $\omega$ que recorre todos los valores de $f$, dando como resultado un espectro continuo. En cambio, la DFT posee una variable $k$ discreta, que recorre solo los valores enteros entre 0 y $N$. Lo mencionado se puede observar en la figura \ref{fig.3a}.
    
    
    \begin{figure}[htb]
    \centering
	%\includegraphics[width=\textwidth]{Img/punto_3_a.png}
	\caption{DTFT de la señal $x[n]$}
	\label{fig.3a}
    \end{figure}
    
    De los gráficos mostrados en la figura \ref{fig.3a}, tiene sentido concluir que la DFT muestrea a la DTFT, utilizando los N valores que se le adjudicaron para realizar el cálculo. En la ecuación (\ref{omega.k}) se aprecia como la DFT discretiza la frecuencia angular, dando como resultado que sobre dicha transformada, la DTFT actúe de envolvente. Sin más preámbulos, no se puede obtener de forma exacta la DTFT a partir de la DFT, al ser una expresión que involucre una variable continua.
    
    %Al comparar los resultados, ecuación \ref{DTFT.R} con ecuación \ref{DFT.R}, se observa que la diferencia recae en la frecuencia. Debido a esto, en la ecuación 
    %\ref{DTFT.R} la frecuencia asociada es la ecuación \ref{omega} que toma valores para todo $f$ dando como resultado un espectro continuo en frecuencia. Esto es un incoveniente a la hora 
    %de realizar el cálculo numérico (por los infinitos valores).Por esto para la \textit{DFT} se discretiza la frecuencia angular dando como resultado la ecuación \ref{omega.k}. Se concluye que es imposible encontrar de forma exacta la \textit{DTFT} a partir de la \textit{DFT}. En otras palabras, se puede decir que 
    %la \textit{DFT} muestrea el espectro continuo de la \textit{DTFT}.

    \subsection{Cálculo de la DFT con distintos puntos}    
    Continuando con el estudio de la DFT, se la calcula esta vez de forma numérica utilizando la \textit{fft}. Para llevar esto a cabo, se genera la señal de la ecuación (\ref{eq.xn}) para los siguientes valores de N: 3, 4 y 8. Los resultados de los tres caso están presentados en la figura \ref{fig.3b}:

    \begin{figure}[htb]
    \centering
    \includegraphics[width=\textwidth]{Img/punto_3_b.png}
    \caption{Espectros generados con 3,4 y 8 puntos.}
    \label{fig.3b}
    \end{figure}

    Se añadió en los gráficos una señal punteada que corresponde al espectro real de la señal $x[n]$.
    Se puede apreciar que los valores que toma la DFT coinciden con el valor de la envolvente en dichos puntos, por lo que seria posible aproximar la DTFT con la DFT.

    \subsection{DTF con zero-padding}
    Se realizo ahora nuevamente el calculo de la DFT de las señales de la seccion anterior pero ahora con el objetivo de estudiar el efecto del \textit{zero-padding}, el cual consiste en rellenar con ceros una señal hasta un numero determinado de puntos. Se tomaron las señales anteriores y se rellenaron con ceros hasta tener 256 puntos para luego calcular su DFT, los resultados obtenidos se muestran en la figura \ref{fig.1c}
    \begin{figure}[htb]
    \centering
    \includegraphics[width=\textwidth]{Img/punto_3_c.png}
    \caption{\textit{fft} con zero-padding hasta 256 puntos para 3,4 y 8 puntos de la señal $x[n]$}
    \label{fig.1c}
    \end{figure}
    Ahora es posible observar de mejor manera, lo comentado en el apartado anterior, como el zero-padding no agrega mas informacion al espectro sino que permite una mejor visualización del mismo, cuando se toman 3 puntos de la señal original, no se llega a tomar la señal completa si no que se recorta, por lo que el espectro si bien un aspecto similar, es mas ancho y de menor amplitud(al ser un cajon de 3 puntos y no de 4 puntos como la señal original), y para los casos de N=4 y N=8 que si se utiliza toda la informacion de la señal $x[n]$ se ve claramente que los valores de la DFT corresponden a muestras de la DTFT.


    \subsection{Calculo exacto de la DTFT}

        Para obtener la DTFT a partir de la \textit{fft}, la cual se puede asociar a un muestreo del espectro
        de la DTFT, mediante la siguiente relacion 

        \begin{equation}
            X[k]=X_{DTFT}\left(  \frac{2 \pi}{N} k \right)
        \end{equation}

        Donde $N$ es el numero de puntos que se utilizan para calcular la DFT.

        El calculo de la DTFT mediante el algoritmo de la \textit{fft} nunca sera exacto. Debido a que la DTFT es continua en frecuencia 
        mientras que la la DFT no. Pero los valores obtenido de la \textit{fft}, pueden ser considerados como muestras del espectro siempre que se 
        cumpla que 

        \begin{enumerate}
            \item Señal sea de duracion finita ($L$).
            \item Utilizar al menos $L$ puntos.
        \end{enumerate}
        


    
    \subsection{codigo para la obtencion de la DTFT}
    Resultado teórico de cada una de las ecuaciones: 

    \subsubsection*{i)}
    \begin{equation}
    x_{1}[n] = \left\{ 
        \begin{array}{ll} 
        n & \mathrm{si\ } 0\leq n \leq 4 \\
        10-n & \mathrm{si\ } 5\leq n \leq 10 \\
        0 & \mathrm{sino\ } \\
        \end{array} 
        \right.
    \end{equation}
    Por tabla se llega al resultado:
    \begin{equation}
    X_{1}(e^{j \omega})=25 sinc^{2}\left(\frac{5 \omega}{2 \pi}\right) e^{- j \omega} 
    \end{equation}

    \subsubsection*{ii)}
    \begin{equation} 
        x_{2}[n] = \left\{ 
        \begin{array}{ll} 
        1 & \mathrm{si\ } -2\leq n \leq +2, \\
        0 & \mathrm{sino\ } \\
        \end{array} 
        \right.
    \end{equation}

    Mediante el uso de tablas de transformada se llego a que el espectro esta dado por:
    \begin{equation}
    X_{2}(e^{j \omega})= \frac{sin(\frac{ \omega 5}{2})}{sin (\frac{\omega}{2})}
    \end{equation}

    \subsubsection*{iii)}
    \begin{equation} 
        x_{3}[n]=(-0,5)^{n}u[n]
    \end{equation}
    Donde nuevamente su espectro se obtuvo por tablas, dando como resultado:
    \begin{equation}
    X(e^{j \omega})=\frac{1}{1+0,5 e^{j \omega}}
    \end{equation}
    Mediante la implementacion del codigo se calculo la DTFT y se obtuvieron las siguiente figuras:
    \begin{figure}[H]
    \centering
    \includegraphics[width=\textwidth]{Img/punto_3_e_1.png}
    \caption{Espectro de la señal $x_{1}[n]$ mediante el algoritmo.}
    \label{fig.3ei}
    \end{figure} 

    \begin{figure}[H]
    \centering
    \includegraphics[width=\textwidth]{Img/punto_3_e_3.png}
    \caption{Espectro de la señal $x_{2}[n]$ mediante el algoritmo.}
    \label{fig.3eii}
    \end{figure}

    \begin{figure}[H]
    \centering
    \includegraphics[width=\textwidth]{Img/punto_3_e_3.png}
    \caption{Espectro de la señal $x_{3}[n]$ mediante el algoritmo.}
    \label{fig.3eiii}
    \end{figure}

    Se puede apreciar que los espectros representados en las figuras \ref{fig.3ei},\ref{fig.3eii} y \ref{fig.3eiii} coinciden de manera aproximada con lo obtenido de manera teorica. 
    
    
\begin{figure}[htb]
\begin{subfigure}{.6\textwidth}
  \centering
  % include first image
  \includegraphics[width=.8\linewidth]{Img/punto_3_b.png}  
  \caption{Put your sub-caption here}
  \label{fig:sub-first}
\end{subfigure}
\begin{subfigure}{.6\textwidth}
  \centering
  % include second image
  \includegraphics[width=\linewidth]{Img/punto_3_c.png}  
  \caption{Put your sub-caption here}
  \label{fig:sub-second}
\end{subfigure}
\caption{Put your caption here}
\label{fig:fig}
\end{figure}



\end{document}