\documentclass[letterpaper]{article}
\usepackage[spanish,es-tabla]{babel}
\usepackage{indentfirst}
\usepackage{float}
\usepackage[utf8]{inputenc}
\usepackage{graphicx}

\begin{document}

\section{Ejercicio 4}
\subsubsection*{Incisos g)}
 Para el cálculo de la CTFT utilizando la DFT se realizan los siguientes pasos:
 \begin{itemize}
     \item Obtención de la señal $x[n]$ muestreando la señal $x(t)$ con una frecuencia $F_s$.
     \item Implementar el algoritmo de la fft a la señal muestreada.
     \item Normalizar el resultado obtenido de la fft dividiendo por $F_s$.
     \item VER - Filtrar un periodo del espectro obtenido de la DFT - VER
 \end{itemize}

 Bajo ninguna condición la estimación de la CTFT puede ser exacta debido a su naturelaza continua y los problemas numericos que esto conlleva. Sin embargo, se puede obtener una 
 estimación precisa si se eligen de manera correcta los siguientes parametros:
 \begin{enumerate}
     \item Frecuencia de muestreo: se debe elegir una $F_s$ que cumpla el teorema de muestreo (ref). Esta condición se ve reflejada en el \textit{aliasing} resultante del espectro. Si bien el espectro siempre va presentar aliasing (debido a que se esta muestreando una señal aperiódica) éste se puede disminuir al utilizar $F_s$ cada vez mayores.
     \item Cantidad de muestras: se debe elegir una cantidad de muestras $N$ mínima que contenga la información de la señal muestreada. A partir de esta idea se desprenden dos situaciones:
     \begin{itemize}
         \item Señal $x[n]$ de duración finita: En este caso si la longitud de la señal es igual a $L$, la cantidad de muestras debe ser $N \geq L$.
         \item Señal $x[n]$ de duración infinita: En este caso la cantidad de muestras $N$ debe ser lo mas representativa posible, es decir mientras mayor sea $N$ mejor sera la aproximación.
     \end{itemize}
\end{enumerate}

Por último, el efecto del método zero-padding tiene como finalidad mejorar la visualización del espectro. (Esto se desarrollo en la sección (ref)).
    
\end{document}