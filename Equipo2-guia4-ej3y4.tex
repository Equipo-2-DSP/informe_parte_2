\documentclass[letterpaper]{article}
\usepackage[spanish,es-tabla]{babel}
\usepackage{indentfirst}
\usepackage{float}
\usepackage[utf8]{inputenc}
\usepackage{graphicx}

\begin{document}
    
    \subsection*{Calculo exacto de la DTFT}

    Para obtener la DTFT a partir de la \textit{fft}, la cual se puede asociar a un muestreo del espectro
    de la DTFT, mediante la siguiente relacion 

    \begin{equation}
        X[k]=X_{DTFT}\left(  \frac{2 \pi}{N} k \right)
    \end{equation}

    Donde $N$ es el numero de puntos que se utilizan para calcular la DFT.

    El calculo de la DTFT mediante el algoritmo de la \textit{fft} nunca sera exacto. Debido a que la DTFT es continua en frecuencia 
    mientras que la la DFT no. Pero los valores obtenido de la \textit{fft}, pueden ser considerados como muestras del espectro siempre que se 
    cumpla que 

    \begin{enumerate}
        \item Señal sea de duracion finita ($L$).
        \item Utilizar al menos $L$ puntos.
    \end{enumerate}

    

\end{document}