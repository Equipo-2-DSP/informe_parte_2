\documentclass[letterpaper]{article}
\usepackage[spanish,es-tabla]{babel}
\usepackage{indentfirst}
\usepackage{float}
\usepackage[utf8]{inputenc}
\usepackage{graphicx}

\begin{document}
\section{Ejercicio 3}
\begin{equation}
    x[n]=\delta[n]+\delta[n-1]+\delta[n-2]+\delta[n-3]
\end{equation}
\subsection*{Inciso a)}
Utilizando la definción de la DTFT:
\begin{equation}
    X(e^{j\omega})=\sum_{n=-\infty}^{\infty}x[n]e^{-j\omega n}
\end{equation}
Donde:
\begin{equation}
    \label{omega}
    \omega=2\pi f
\end{equation}
Llegamos a:
\begin{equation}
    \label{DTFT.R}
    X(e^{j\omega})=1+e^{-j\omega}+e^{-2j\omega}+e^{-3j\omega}
\end{equation}
Para el cálculo de la DFT utilizamos la ecuación:
\begin{equation}
    X[k]=\sum_{n=0}^{N-1}x[n]e^{-j\omega_k n}
\end{equation}
Donde:
\begin{equation}
    \label{omega.k}
    \omega_k=\frac{2\pi k}{N}
\end{equation}
Obteniendo:
\begin{equation}
    \label{DFT.R}
    X[k]=1+e^{\frac{-j2\pi k }{4}}+e^{\frac{-j4\pi k}{4}}+e^{\frac{-j6\pi k}{4}}
\end{equation}
Al comparar los resultados, ecuación \ref{DTFT.R} con ecuación \ref{DFT.R}, se observa que la diferencia recae en la frecuencia. Debido a esto, en la ecuación 
\ref{DTFT.R} la frecuencia asociada es la ecuación \ref{omega} que toma valores para todo $f$ dando como resultado un espectro continuo en frecuencia. Esto es un incoveniente a la hora 
de realizar el cálculo numérico (por los infinitos valores).Por esto para la \textit{DFT} se discretiza la frecuencia angular dando como resultado la ecuación \ref{omega.k}. Se concluye que es imposible encontrar de forma exacta la \textit{DTFT} a partir de la \textit{DFT}. En otras palabras, se puede decir que 
la \textit{DFT} muestrea el espectro continuo de la \textit{DTFT}.
\end{document}