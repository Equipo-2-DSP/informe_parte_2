\documentclass[11pt,a4paper]{article}
\usepackage[utf8]{inputenc}
\usepackage[spanish,es-tabla]{babel}
\usepackage{amsmath}
\usepackage{amsfonts}
\usepackage{amssymb}
\usepackage{graphicx}
\usepackage{float}
\usepackage{hyperref}
%\usepackage{subfigure} 
\usepackage{amsmath}
\usepackage{amsfonts}
\usepackage[usenames]{color}
\usepackage{subcaption}
\usepackage[left=3cm,right=2cm,top=3cm,bottom=3cm]{geometry}
\author{Gatica, Isaias \\ Martin, Santiago \\ Saez, Lautaro Ándres \\ Vidman, Xavier Harry}
\title{Procesamiento digital de señales: estimación del espectro de una señal aperiódica}
\date{}
\begin{document}
\maketitle
\section{Análisis de una señal aperiódica y discreta}
En esta sección se analiza el uso de la DFT para estimar el espectro de una señal aperiódica y discreta. Para ello, se considera en primera instancia la siguiente señal:

\begin{equation}
    x[n]=\delta[n]+\delta[n-1]+\delta[n-2]+\delta[n-3]
    \label{eq.xn}
\end{equation}

Para comenzar el estudio se calcula la DTFT y la DFT de la ecuación (\ref{eq.xn}), a fin de realizar una comparación entre los resultados, y poder concluir si es posible utilizar a la DFT para obtener de forma exacta o aproximada la DTFT de una señal aperiódica.

    \subsection{Cálculo de la DTFT y la DFT}

    Utilizando la definción de la DTFT\footnote{Dimitris Manolakiv and Vinay Ingle - ``Applied Digital Signal Processing'' 5th Edition.}:
    \begin{equation}
        X(e^{j\omega})=\sum_{n=-\infty}^{\infty}x[n]e^{-j\omega n}
    \end{equation}
    Donde:
    \begin{equation}
        \label{omega}
        \omega=2\pi f
    \end{equation}
    Se llega a que la DTFT de la ecuación \ref{eq.xn} es:
    \begin{equation}
        X(e^{j\omega})=1+e^{-j\omega}+e^{-2j\omega}+e^{-3j\omega}
        \label{DTFT.R}
    \end{equation}
    Siguiendo con el análisis, se calcula la DFT utilizando la siguiente definición\footnote{Idem 1.}:
    \begin{equation}
        X[k]=\sum_{n=0}^{N-1}x[n]e^{-j\omega_k n} \qquad , 0 \leq k \leq 7
    \end{equation}
    Donde:
    \begin{equation}
        \label{omega.k}
        \omega_k=\frac{2\pi k}{N}
    \end{equation}
    Obteniendo como resultado:
    \begin{equation}
        \label{DFT.R}
        X[k]=1 + e^{-j \frac{\pi}{4} k} + e^{-j \frac{\pi}{2} k} + e^{-j \frac{3 \pi}{4} k}
    \end{equation}
    
    Al cotejar los resultados obtenidos, se observa que las expresiones son muy similares. La diferencia recae en las frecuencias de las exponenciales. La DTFT expresada en la ecuación (\ref{DTFT.R}) tiene como variable una frecuencia $\omega$ que recorre todos los valores de $f$, dando como resultado un espectro continuo. En cambio, la DFT posee una variable $k$ discreta, que recorre solo los valores enteros entre 0 y $N$. Lo mencionado se puede observar en la figura \ref{fig.3a}.
    
    
    \begin{figure}[htb]
    \centering
	%\includegraphics[width=\textwidth]{Img/punto_3_a.png}
	\caption{DTFT de la señal $x[n]$}
	\label{fig.3a}
    \end{figure}
    
    De los gráficos mostrados en la figura \ref{fig.3a}, tiene sentido concluir que la DFT muestrea a la DTFT, utilizando los N valores que se le adjudicaron para realizar el cálculo. En la ecuación (\ref{omega.k}) se aprecia como la DFT discretiza la frecuencia angular, dando como resultado que sobre dicha transformada, la DTFT actúe de envolvente. Sin más preámbulos, no se puede obtener de forma exacta la DTFT a partir de la DFT, al ser una expresión que involucre una variable continua.
    
    %Al comparar los resultados, ecuación \ref{DTFT.R} con ecuación \ref{DFT.R}, se observa que la diferencia recae en la frecuencia. Debido a esto, en la ecuación 
    %\ref{DTFT.R} la frecuencia asociada es la ecuación \ref{omega} que toma valores para todo $f$ dando como resultado un espectro continuo en frecuencia. Esto es un incoveniente a la hora 
    %de realizar el cálculo numérico (por los infinitos valores).Por esto para la \textit{DFT} se discretiza la frecuencia angular dando como resultado la ecuación \ref{omega.k}. Se concluye que es imposible encontrar de forma exacta la \textit{DTFT} a partir de la \textit{DFT}. En otras palabras, se puede decir que 
    %la \textit{DFT} muestrea el espectro continuo de la \textit{DTFT}.

    \subsection{Cálculo de la DFT con distintos puntos}    
    Continuando con el estudio de la DFT, se realiza su cálculo de forma numérica utilizando la \textit{fft}. Para llevar esto a cabo, se genera la señal de la ecuación (\ref{eq.xn}) para los siguientes valores de N: 3, 4 y 8. Los resultados de los tres casos están presentados en la figura \ref{fig.3b}, junto con una señal punteada que corresponde al espectro real de la señal $x[n]$:

    \begin{figure}[htb]
    \centering
    \includegraphics[width=\textwidth]{Img/punto_3_b.png} 
    \caption{Espectros generados con N=3, N=4 y N=8.}
    \label{fig.3b}
    \end{figure}
    
    Analizando los resultados de la figura \ref{fig.3b}, se puede ver que tomar 3 puntos de la señal original difiere con los resultados de la envolvente, y al continuar realizando pruebas con otros valores de N, se observa que al tomar valores de $N$ por debajo de 4, el espectro obtenido a través de la \textit{fft} posee valores diferentes al original. Esto sucede porque se está recortando la señal $x[n]$, y por ende, perdiendo información, lo que termina en un espectro erróneo.
    Asimismo, los valores de la DFT coinciden con el valor de la envolvente en los casos que $N=4$ y $N=8$. Extrapolando estos resultados, se puede decir que para cualquier valor de $N$ mayor a 4, los valores calculados por la \textit{fft} coinciden con los valores de la envolvente.

    \subsection{DFT con zero-padding}
    En esta sección, se realiza el cálculo de la DFT de la señal (\ref{eq.xn}), pero esta vez, con el objetivo de estudiar el efecto del \textit{zero-padding}. Dicho efecto consiste en rellenar con ceros una señal hasta un número determinado de puntos. En este caso, se tomaron las señales anteriores (valores de N: 3, 4 y 8) y se rellenaron con ceros hasta tener 256 puntos, para luego calcular su DFT mediante la \textit{fft}. Los resultados obtenidos se muestran en la figura \ref{fig.1c}.
    
    \begin{figure}[H]
    \centering
    \includegraphics[width=\textwidth]{Img/punto_3_c.png}
    \caption{\textit{fft} con zero-padding hasta 256 puntos para 3,4 y 8 puntos de la señal $x[n]$}
    \label{fig.1c}
    \end{figure}
    
    Al observar la figura \ref{fig.1c}, se concluye que el método \textit{zero-padding} no agrega más información al espectro, sino que solo permite una mejor visualización del espectro real de señal. No obstante, por más que se utilice el \textit{zero-paddding}, si no se realiza con todos los valores distintos de cero de la señal $x[n]$, la aproximación seguirá siendo incorrecta.
    
    %Ahora es posible observar de mejor manera, lo comentado en el apartado anterior, como el zero-padding no agrega mas información al espectro sino que permite una mejor visualización del mismo, cuando se toman 3 puntos de la señal original, no se llega a tomar la señal completa si no que se recorta, por lo que el espectro si bien un aspecto similar, es mas ancho y de menor amplitud (al ser un cajon de 3 puntos y no de 4 puntos como la señal original), y para los casos de N=4 y N=8 que si se utiliza toda la información de la señal $x[n]$ se ve claramente que los valores de la DFT corresponden a muestras de la DTFT.


    \subsection{Aproximación de la DTFT utilizando la \textit{fft}}
        En el siguiente apartado se concluye sobre las condiciones necesarias que debe tener una señal aperiódica, para lograr aproximar de una manera correcta el cálculo de su DTFT con el algoritmo \textit{fft}. 
        
        
        De los análisis anteriores se sabe que la DFT muestrea al espectro continuo de una señal $x[n]$ aperiódica. Por lo tanto, se afirma que no existe una manera numérica de calcular la DTFT para señales aperiódicas, ya que se necesitaría almacenar infinitos valores para realizarlo. La siguiente ecuación expresa la relación que hay entre la frecuencia discretizada de la DFT con la frecuencia de la DTFT: 
        
        \begin{equation}
            X[k]=X_{DTFT}\left(  \frac{2 \pi}{N} k \right)
        \end{equation}

        Donde $N$ es el número de puntos que se utilizan para calcular la DFT.

        No obstante, hay ciertas condiciones que se deben cumplir para lograr una buena aproximación, y para que los valores obtenidos de la \textit{fft} sean considerados muestras del espectro real de la señal. Las condiciones son las siguientes:

        \begin{enumerate}
            \item La señal debe ser de duración finita ($L$).
            \item Al realizar el cálculo numérico, utilizar al menos $L$ puntos.
        \end{enumerate}
    
    \subsection{Código para la obtención de la DTFT:}
    En esta sección se utiliza el código adjunto a este informe para calcular la DTFT de tres señales aperiódicas discretas. En principio, se calcula la DTFT de forma teórica. Luego, se muestra en gráficos el resultado obtenido de forma numérica.
    

    \subsubsection*{i) Primera señal:}

    \begin{equation}
    x_{1}[n] = \left\{ 
        \begin{array}{ll} 
        n & \mathrm{si\ } 0\leq n \leq 4 \\
        10-n & \mathrm{si\ } 5\leq n \leq 10 \\
        0 & \mathrm{sino\ } \\
        \end{array} 
        \right.
    \end{equation}
    
    
    Resultado teórico:
    
    
    \begin{equation}
    X_{1}(e^{j \omega})=25 sinc^{2}\left(\frac{5 \omega}{2 \pi}\right) e^{- j \omega} 
    \end{equation}

    \subsubsection*{ii) Segunda señal:}

    \begin{equation} 
        x_{2}[n] = \left\{ 
        \begin{array}{ll} 
        1 & \mathrm{si\ } -2\leq n \leq +2, \\
        0 & \mathrm{sino\ } \\
        \end{array} 
        \right.
    \end{equation}

    Resultado teórico:
    
    \begin{equation}
    X_{2}(e^{j \omega})= \frac{sin(\frac{ \omega 5}{2})}{sin (\frac{\omega}{2})}
    \end{equation}

    \subsubsection*{iii) Tercera señal:}

    \begin{equation} 
        x_{3}[n]=(-0,5)^{n}u[n]
    \end{equation}
    
    Resultado teórico:
    
    \begin{equation}
    X(e^{j \omega})=\frac{1}{1+0,5 e^{j \omega}}
    \end{equation}
    
    A continuación, se pueden observar los resultados de la DTFT de forma numérica en las siguientes figuras:
    
    \begin{figure}[H]
    \centering
    \includegraphics[width=\textwidth]{Img/punto_3_e_1.png}
    \caption{espectro de la señal $x_{1}[n]$ mediante el algoritmo.}
    \label{fig.3ei}
    \end{figure} 


    \begin{figure}[H]
    \centering
    \includegraphics[width=\textwidth]{Img/punto_3_e_3.png}
    \caption{espectro de la señal $x_{2}[n]$ mediante el algoritmo.}
    \label{fig.3eii}
    \end{figure}


    \begin{figure}[H]
    \centering
    \includegraphics[width=\textwidth]{Img/punto_3_e_3.png}
    \caption{espectro de la señal $x_{3}[n]$ mediante el algoritmo.}
    \label{fig.3eiii}
    \end{figure}

    Se puede apreciar que los espectros representados en las figuras \ref{fig.3ei},\ref{fig.3eii} y \ref{fig.3eiii} coinciden de manera aproximada con lo obtenido de manera teórica. 
    
    
%\begin{figure}[htb]
%\begin{subfigure}{.6\textwidth}
%  \centering
  % include first image
%  \includegraphics[width=.8\linewidth]{Img/punto_3_b.png}  
%  \caption{Put your sub-caption here}
%  \label{fig:sub-first}
%\end{subfigure}
%\begin{subfigure}{.6\textwidth}
%  \centering
  % include second image
%  \includegraphics[width=\linewidth]{Img/punto_3_c.png}  
%  \caption{Put your sub-caption here}
%  \label{fig:sub-second}
%\end{subfigure}
%\caption{Put your caption here}
%\label{fig:fig}
%\end{figure}



\end{document}